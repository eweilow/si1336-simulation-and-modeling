\documentclass[11pt]{article}
\usepackage[cm]{fullpage}
\usepackage{graphicx}
\usepackage{caption}
\usepackage{subcaption}
\usepackage[section]{placeins}
\usepackage{float}

\title{S1336 - Project 1}%replace with the appropriate homework number
\author{Erik Weilow} %if necessary, replace with your course title

\newcommand{\triplefigure}[3]{
\begin{figure}[H]
  \centering
  \begin{minipage}{0.3\textwidth}
    \centering
    \includegraphics[width=\textwidth]{#1}
  \end{minipage}
  \begin{minipage}{0.3\textwidth}
    \centering
    \includegraphics[width=\textwidth]{#2}
  \end{minipage}
  \begin{minipage}{0.3\textwidth}
    \centering
    \includegraphics[width=\textwidth]{#3}
  \end{minipage}
\end{figure}
}
\newcommand{\singlefigure}[1]{
\begin{figure}[H]
  \centering
  \begin{minipage}{0.4\textwidth}
    \centering
    \includegraphics[width=\textwidth]{#1}
  \end{minipage}
\end{figure}
}

\begin{document}
\maketitle
\newpage

\section*{1.1}

\subsection*{Pendulums vs harmonic oscillators}
\triplefigure{./plots/1_1/comparison_1.png}{./plots/1_1/comparison_2.png}{./plots/1_1/comparison_3.png}
In this first part, it can be seen that a numerical solution for a pendulum has longer period than for that of a harmonic oscillator.
Increasing the initial angle increases the amplitude of oscillations.

\subsection*{Pendulums: general solution}
\triplefigure{./plots/1_1/sol_pendulum_1.png}{./plots/1_1/sol_pendulum_2.png}{./plots/1_1/sol_pendulum_3.png}
It can be seen that the solution doesn't scale linearly with increased starting angle. 

\subsection*{Harmonic oscillators: general solution}
\triplefigure{./plots/1_1/sol_harmonic_1.png}{./plots/1_1/sol_harmonic_2.png}{./plots/1_1/sol_harmonic_3.png}
More or less the same result as for a pendulum, though with a slightly shorter period as previously shown. 

\subsection*{Rolling mean of energy for pendulum}
\triplefigure{./plots/1_1/rollingMean_1.png}{./plots/1_1/rollingMean_2.png}{./plots/1_1/rollingMean_3.png}
We can't even see the simulation using Euler in this plot - it shot right up within mere seconds.
Runge Kutta becomes "good" on timescales on the order of 1000 seconds for $\Delta t \approx 0.01$.
For longer timescales, we need a smaller $\Delta t$.

Both Euler-Cromer and Velocity Verlet have energies that oscillate fast enough to cause aliasing effects (hence why the rolling average of those are a solid block of color).

\subsection*{Rolling mean of energy for harmonic oscillator}
\triplefigure{./plots/1_1/rollingMean_harmonic_1.png}{./plots/1_1/rollingMean_harmonic_2.png}{./plots/1_1/rollingMean_harmonic_3.png}
For the harmonic oscillator, we get basically the same results as for the pendulum but the energies oscillate less for Euler-Cromer and Velocity Verlet.

\subsection*{Harmonic oscillators: Numerical vs analytic}
\triplefigure{./plots/1_1/comparison_numanalytic_1.png}{./plots/1_1/comparison_numanalytic_2.png}{./plots/1_1/comparison_numanalytic_3.png}
Using Verlet integration: sees a drift in period, but the numerical integration is stable.

\section*{1.2}
\singlefigure{./plots/1_2/period.png}
Harmonic oscillator has constant period, smaller than pendulum.
The perturbation series does not approximate the harmonic oscillator well.

\section*{1.3}

\subsection*{Dampened harmonic oscillator}
\triplefigure{./plots/1_3/dampened_dx.png}{./plots/1_3/dampened_E.png}{./plots/1_3/dampened_x.png}
Energy decreases over time with dampening.

\subsection*{Relaxation time as function of $\gamma$}
\singlefigure{./plots/1_3/study.png}
Fitting $f(t) = A e^{-Bt}$ to local peaks of simulated oscillations.

\subsection*{Minimum x as function of $\gamma$}
\singlefigure{./plots/1_3/study_2.png}
Energy decreases over time with dampening.

\section*{1.4}
\subsection*{Phase space portrait of dampened pendulum}
\singlefigure{./plots/1_4/study.png}

\section*{1.5}
\subsection*{Energy average}
\singlefigure{./plots/1_5/study.png}
Both are energy preserving over long time, even for silly large time steps.

\subsection*{Energy average}
\singlefigure{./plots/1_5/study2.png}
Leapfrog produces different trajectory.

% ![](./plots/1_1/comparison_1.png)
% ![](./plots/1_1/comparison_2.png)
% ![](./plots/1_1/comparison_3.png)
\end{document}