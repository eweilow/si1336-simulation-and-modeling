\documentclass[11pt]{article}
\usepackage[cm]{fullpage}
\usepackage{graphicx}
\usepackage{caption}
\usepackage{subcaption}
\usepackage[section]{placeins}
\usepackage{float}
\usepackage{amsmath}
\usepackage{multicol}

\setlength{\columnsep}{1cm}

\title{S1336 - Project 5}
\author{Erik Weilow}

\newcommand{\triplefigure}[3]{
\begin{figure}[H]
  \centering
  \begin{minipage}{0.3\textwidth}
    \centering
    \includegraphics[width=\textwidth]{#1}
  \end{minipage}
  \begin{minipage}{0.3\textwidth}
    \centering
    \includegraphics[width=\textwidth]{#2}
  \end{minipage}
  \begin{minipage}{0.3\textwidth}
    \centering
    \includegraphics[width=\textwidth]{#3}
  \end{minipage}
\end{figure}
}
\newcommand{\doublefigure}[2]{
\begin{figure}[H]
  \centering
  \begin{minipage}{0.45\textwidth}
    \centering
    \includegraphics[width=\textwidth]{#1}
  \end{minipage}
  \begin{minipage}{0.45\textwidth}
    \centering
    \includegraphics[width=\textwidth]{#2}
  \end{minipage}
\end{figure}
}
\newcommand{\singlefigure}[1]{
\begin{figure}[H]
  \centering
  \begin{minipage}{0.4\textwidth}
    \centering
    \includegraphics[width=\textwidth]{#1}
  \end{minipage}
\end{figure}
}
\newcommand{\singlewiderfigure}[1]{
\begin{figure}[H]
  \centering
  \begin{minipage}{0.6\textwidth}
    \centering
    \includegraphics[width=\textwidth]{#1}
  \end{minipage}
\end{figure}
}

\begin{document}
\maketitle
\newpage

\section*{5.1}

We are to use the Metropolis method to calculate
$$
\langle x \rangle = \frac{\int_0^\infty{x e^{-x} dx}}{\int_0^\infty{e^{-x} dx}}
$$
for different values $\delta$ controlling the randomness of our algorithm. The first modification that has to be done is to create the modified distribution
$$
P(x) = H(x) e^{-x}
$$
where $H(x)$ is the Heaviside function. This lets us expand the integrals to the equivalent form
$$
\langle x \rangle = \frac{\int_{-\infty}^\infty{x H(x) e^{-x} dx}}{\int_{-\infty}^\infty{H(x)e^{-x} dx}}
$$
Otherwise the model statistically produces very few positive values of x and thus we fail to compute the desired $\langle x \rangle$.

The following are plots for different choices of $\delta$
\doublefigure{./plots/5_1/mean.png}{./plots/5_1/delta.png}

It would seem that larger $\delta$ tends to be slightly better than smaller, but after 2 million generated points they are all very close in terms of $\Delta = \frac{\sigma}{N}$.

We can also look at the difference to the theoretical answer, as well as the difference between $\Delta$ and $|\langle x \rangle - 1|$.
\doublefigure{./plots/5_1/difference.png}{./plots/5_1/diff.png}

This further shows that larger $\delta$ tends to be better, up to a point.

\section*{5.2}
\singlewiderfigure{./plots/5_2/acceptance.png}
The above is a plot of the acceptance ratio versus step size. The acceptance ratio around 70 percent for step size of 0.1 and tends to 0 for step sizes between 1 and 10.


\doublefigure{./plots/5_2/convergence.png}{./plots/5_2/cV.png}

We see here that the convergence is $\langle E \rangle = -37.5$ and $cV = 10$ for a step size $>0.1$. Calling the region around $10^{-4}$ convergence would be weird, as $cV$ approaches 0. This seems to me ofysikaliskt.

\section*{A note}
\textit{For the remainder of this report, a step size of 0.1 was used.}

\section*{5.3}
I wanted to look into a more rigorous approach to the determination of collective behaviour, rather than just animating a single simulation for different temperatures.
One such metric is the average distance between particles as shown in the following figure.
\singlewiderfigure{./plots/5_3/acceptance.png}
We can see that the particle distance seem to peak around a temperature of approximately $T\approx 1$. 

The same simulation data is plotted on a narrower and linear temperature range in the following figure.
\singlewiderfigure{./plots/5_3/small_range.png}

An addition that also could be nice to look at is the average velocities for the particles.

\section*{5.4}
\doublefigure{./plots/5_4/energy.png}{./plots/5_4/heat.png}
In the two above plots we can see total energy and the heat capacity of the 20 particle system as function of temperature.
Energy seem to increase with temperature (reasonable) while heat capacity decreases (maybe reasonable?).
It's clear from thermodynamics that increasing temperature should result in higher energy, but heat capacity decreasing is counterintuitive.
My thinking is that this has to do with the average distance plot seen in the previous section. The density is lowest around $T=1$ and thus heat capacity is less?


\end{document}