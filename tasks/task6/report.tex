\documentclass[11pt]{article}
\usepackage[cm]{fullpage}
\usepackage{graphicx}
\usepackage{caption}
\usepackage{subcaption}
\usepackage{titlesec}
\usepackage[section]{placeins}
\usepackage{float}
\usepackage{amsmath}
\usepackage{multicol}

\setlength{\columnsep}{1cm}

\title{S1336 - Project 6}
\author{Erik Weilow}

\newcommand{\triplefigure}[3]{
\begin{figure}[H]
  \centering
  \begin{minipage}{0.3\textwidth}
    \centering
    \includegraphics[width=\textwidth]{#1}
  \end{minipage}
  \begin{minipage}{0.3\textwidth}
    \centering
    \includegraphics[width=\textwidth]{#2}
  \end{minipage}
  \begin{minipage}{0.3\textwidth}
    \centering
    \includegraphics[width=\textwidth]{#3}
  \end{minipage}
\end{figure}
}
\newcommand{\doublefigure}[2]{
\begin{figure}[H]
  \centering
  \begin{minipage}{0.4\textwidth}
    \centering
    \includegraphics[width=\textwidth]{#1}
  \end{minipage}
  \begin{minipage}{0.4\textwidth}
    \centering
    \includegraphics[width=\textwidth]{#2}
  \end{minipage}
\end{figure}
}
\newcommand{\singlefigure}[1]{
\begin{figure}[H]
  \centering
  \begin{minipage}{0.4\textwidth}
    \centering
    \includegraphics[width=\textwidth]{#1}
  \end{minipage}
\end{figure}
}
\newcommand{\singlewiderfigure}[1]{
\begin{figure}[H]
  \centering
  \begin{minipage}{0.6\textwidth}
    \centering
    \includegraphics[width=\textwidth]{#1}
  \end{minipage}
\end{figure}
}

\newcommand{\task}[2]{
  \subsection*{Subtask #1)}
  \begin{tabular}{|p{0.9\textwidth}}
    #2\\
  \end{tabular}
}

\newcommand{\subtask}[1]{
  \subsubsection*{#1}
}

\begin{document}
\maketitle
\newpage

\section*{10.10 Numerical solution of the potential within a rectangular region}
\task{a}{
  Determine the potential $V(x, y)$ in a square region with linear dimension $L = 10$. 
  The boundary of the square is at a potential $V = 10$. 
  Choose the grid size $\Delta x = \Delta y = 1$
}
\subtask{Guess the exact form of $V(x, y)$}
A guess is $V(x,y) = 10$. This is reasonable as all the boundaries are $V=10$ and the algorithm algorithm makes each point the average of the points around it.
For a small grid ($\Delta x = \Delta y = 5$), the center point would be instantly equal to 10.

\subtask{How many iterations are necessary to achieve 1\% accuracy?}
It takes about 56 relaxations to achieve 1\% accuracy, producing the shape seen in the below plots.
\doublefigure{./plots/1010/a_1.png}{./plots/1010/a_1_eq.png}

\subtask{Decrease the grid size by a factor of two, and determine the number of iterations that are now necessary to achieve 1\% accuracy.}
With a grid size $\Delta x = \Delta y = 0.5$, it takes instead closer to 225 relaxations to achieve the same accuracy. 
\singlefigure{./plots/1010/a_2.png}
For these two simple cases, we can see that the amount of relaxations could potentially follow
$$
n_{relaxations} \propto \frac{1}{\Delta x \Delta y} \propto \left( n_{gridpoints} \right)^2
$$

This can be shown via the following loglog diagram showing the necessary relaxations for different grid sizes,
where the curve $0.56 n^2$ can be fit to the computed data.
\singlefigure{./plots/1010/a_1_dependency.png}

\task{b}{
  Consider the same geometry as in part (a), but set the initial potential at the interior sites
  equal to zero except for the center site whose potential is set equal to four.
}
\subtask{Does the potential distribution evolve to the same values as in part (a)?}
It does evolve to the same values as in part (a), but does so much slower with around 102 relaxations necessary.
\singlefigure{./plots/1010/b_1.png}

\subtask{What is the effect of a poor initial guess?}
The result is really only that the convergence is slower.

\subtask{Are the final results independent of your initial guess?}
The final results should be independent of the guess. Since each point in every relaxation step is set to the average of the 4 surrounding points, there will only be no change if all points in the grid fulfill $V(x,y) = 10$.

\task{c}{
  Now modify the boundary conditions so that the value of the potential at the four sides is 5, 10, 5, and 10, respectively.
  (Start with a reasonable guess for the initial values of the potential at the interior sites and iterate until 1\% accuracy is obtained.)
}
\subtask{Sketch the equipotential surfaces.}
By these boundary conditions, we get a saddle as seen in the below figure (and equipotential surfaces plot).
\doublefigure{./plots/1010/c_1.png}{./plots/1010/c_1_eq.png}

\subtask{What happens if the potential is 10 on three sides and 0 on the fourth?}
Now we get a sort of rounded slope, that tends toward $V=0$ for $y = 0$.
\doublefigure{./plots/1010/c_2.png}{./plots/1010/c_2_eq.png}

\section*{10.11 Gauss-Seidel relaxation}
\task{a}{
  Use a modification of the program used in 10.10 so that the potential at each site is updated sequentially. 
  In this way the new potential of the next site is computed using the most recently computed values of its nearest neighbor potentials.
}
\subtask{Are your results better, worse, or about the same as for the simple relaxation method?}
\task{b}{
  Imagine coloring the alternate sites of a grid red and black, so that the grid resembles a checkerboard.
  Modify the program so that all the red sites are updated first, and then all the black sites are updated.
}
\subtask{Do your results converge any more quickly than in part (a)?}

\section*{10.26 The multigrid method}
\task{a}{
  Write a program that implements the multigrid method using Gauss-Seidel relaxation on a checkerboard lattice. 
  Test your program on a $4×4$ grid whose boundary sites are all equal to unity, and whose initial internal sites are set to zero.
}
\task{b}{
  The exact solution for part (a) gives a potential of unity at each point.
}
\subtask{How many relaxation steps does it take to reach unity within 0.1\% at every site by simply using the $4×4$ grid?}
\subtask{How many steps does it take if you use one coarse grid and continue until the coarse grid values are within 0.1\% of unity?}
\subtask{Is it necessary to carry out any fine grid relaxation steps to reach the desired accuracy on the fine grid?}
\subtask{Next start with the coarsest scale, which is just one site. How many relaxation steps does it take now?}

\task{c}{
  Repeat part (b), but change the boundary so that one side of the boundary is held at a potential of 0.5.
}
\subtask{Experiment with different sequences of prolongation, restriction, and relaxation.}

\end{document}