\documentclass[11pt]{article}
\usepackage[cm]{fullpage}
\usepackage{graphicx}
\usepackage{caption}
\usepackage{subcaption}
\usepackage[section]{placeins}
\usepackage{float}

\title{S1336 - Project 1}%replace with the appropriate homework number
\author{Erik Weilow} %if necessary, replace with your course title

\newcommand{\triplefigure}[3]{
\begin{figure}[H]
  \centering
  \begin{minipage}{0.3\linewidth}
    \centering
    \includegraphics[width=\textwidth]{#1}
  \end{minipage}
  \begin{minipage}{0.3\linewidth}
    \centering
    \includegraphics[width=\textwidth]{#2}
  \end{minipage}
  \begin{minipage}{0.3\linewidth}
    \centering
    \includegraphics[width=\textwidth]{#3}
  \end{minipage}
\end{figure}
}

\begin{document}
\maketitle

\section{1.1}

\subsection{Pendulums vs harmonic oscillators}
\triplefigure{./plots/1_1/comparison_1.png}{./plots/1_1/comparison_2.png}{./plots/1_1/comparison_3.png}
In this first part, it can be seen that a numerical solution for a pendulum has longer period than for that of a harmonic oscillator.

\subsection{Pendulums: general solution}
\triplefigure{./plots/1_1/sol_pendulum_1.png}{./plots/1_1/sol_pendulum_2.png}{./plots/1_1/sol_pendulum_3.png}
For a pendulum, the solution doesn't scale linearly with increased starting angle. 

\subsection{Harmonic oscillators: general solution}
\triplefigure{./plots/1_1/sol_harmonic_1.png}{./plots/1_1/sol_harmonic_2.png}{./plots/1_1/sol_harmonic_3.png}
More or less the same result as for a pendulum, though with a slightly shorter period as previously shown. 

\subsection{Rolling mean of energy for pendulum}
\triplefigure{./plots/1_1/rollingMean_1.png}{./plots/1_1/rollingMean_2.png}{./plots/1_1/rollingMean_3.png}
More or less the same result as for a pendulum, though with a slightly shorter period as previously shown. 

\subsection{Rolling mean of energy for harmonic oscillator}
\triplefigure{./plots/1_1/rollingMean_harmonic_1.png}{./plots/1_1/rollingMean_harmonic_2.png}{./plots/1_1/rollingMean_harmonic_3.png}
Same result as for pendulum, expected since period is almost same.


% ![](./plots/1_1/comparison_1.png)
% ![](./plots/1_1/comparison_2.png)
% ![](./plots/1_1/comparison_3.png)
\end{document}