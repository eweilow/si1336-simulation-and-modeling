\documentclass[11pt]{article}
\usepackage[cm]{fullpage}
\usepackage{graphicx}
\usepackage{caption}
\usepackage{subcaption}
\usepackage[section]{placeins}
\usepackage{float}
\usepackage{amsmath}
\usepackage{multicol}
\setlength{\columnsep}{1cm}

\title{S1336 - Project 1}%replace with the appropriate homework number
\author{Erik Weilow} %if necessary, replace with your course title

\newcommand{\triplefigure}[3]{
\begin{figure}[H]
  \centering
  \begin{minipage}{0.3\textwidth}
    \centering
    \includegraphics[width=\textwidth]{#1}
  \end{minipage}
  \begin{minipage}{0.3\textwidth}
    \centering
    \includegraphics[width=\textwidth]{#2}
  \end{minipage}
  \begin{minipage}{0.3\textwidth}
    \centering
    \includegraphics[width=\textwidth]{#3}
  \end{minipage}
\end{figure}
}
\newcommand{\singlefigure}[1]{
\begin{figure}[H]
  \centering
  \begin{minipage}{0.4\textwidth}
    \centering
    \includegraphics[width=\textwidth]{#1}
  \end{minipage}
\end{figure}
}

\begin{document}
\maketitle
\newpage

\section*{1.1}

\subsection*{Pendulums vs harmonic oscillators}
\triplefigure{./plots/1_1/comparison_1.png}{./plots/1_1/comparison_2.png}{./plots/1_1/comparison_3.png}
In this first part, it can be seen that a numerical solution for a pendulum has longer period than for that of a harmonic oscillator.
Increasing the initial angle increases the amplitude of oscillations.

\subsection*{Pendulums: general solution}
\triplefigure{./plots/1_1/sol_pendulum_1.png}{./plots/1_1/sol_pendulum_2.png}{./plots/1_1/sol_pendulum_3.png}
It can be seen that the solution doesn't scale linearly with increased starting angle. 

\subsection*{Harmonic oscillators: general solution}
\triplefigure{./plots/1_1/sol_harmonic_1.png}{./plots/1_1/sol_harmonic_2.png}{./plots/1_1/sol_harmonic_3.png}
More or less the same result as for a pendulum, though with a slightly shorter period as previously shown. 

\subsection*{Rolling mean of energy for pendulum}
\triplefigure{./plots/1_1/rollingMean_1.png}{./plots/1_1/rollingMean_2.png}{./plots/1_1/rollingMean_3.png}
We can't even see the simulation using Euler in this plot - it shot right up within mere seconds.
Runge Kutta becomes "good" on timescales on the order of 1000 seconds for $\Delta t \approx 0.01$.
For longer timescales, we need a smaller $\Delta t$.

Both Euler-Cromer and Velocity Verlet have energies that oscillate fast enough to cause aliasing effects (hence why the rolling average of those are a solid block of color).

\subsection*{Rolling mean of energy for harmonic oscillator}
\triplefigure{./plots/1_1/rollingMean_harmonic_1.png}{./plots/1_1/rollingMean_harmonic_2.png}{./plots/1_1/rollingMean_harmonic_3.png}
For the harmonic oscillator, we get basically the same results as for the pendulum but the energies oscillate less for Euler-Cromer and Velocity Verlet.

\subsection*{Harmonic oscillators: Numerical vs analytic}
\triplefigure{./plots/1_1/comparison_numanalytic_1.png}{./plots/1_1/comparison_numanalytic_2.png}{./plots/1_1/comparison_numanalytic_3.png}
In these plot, the ratio between the analytic and numerical solution is plotted over time:
$$
\frac{1 + \|\text{analytic}(t)\|}{1 + \|\text{numerical}(t)\|}
$$
Using Velocity-Verlet integration we can see aliasing effects (indicative of inequal periods), but the energy is stable as shown previously, so the numerical integration is still stable.

\section*{1.2}
\singlefigure{./plots/1_2/period.png}
Harmonic oscillator has constant period, smaller than pendulum for all but the minimal initial values.
The perturbation series gives a good match for the period of a harmonic oscillator, even if stopping at the sixth power.

\section*{1.3}

\subsection*{Dampened harmonic oscillator}
\triplefigure{./plots/1_3/dampened_dx.png}{./plots/1_3/dampened_E.png}{./plots/1_3/dampened_x.png}
The middle plot shows energy as function of time, in which it's clear that the dampening also decreases energy of the oscillator over time.

From the leftmost figure we see that the velocity of the dampened oscillator decreases with time. A larger $\gamma$ results in shorter relaxation time.

The rightmost plot shows the position of the dampened oscillator over time and it's clear from there as well that higher $\gamma$ results in shorter relaxation time.

A thing to note is that from the selected values of $\gamma$ as well as the chosen initial conditions, the oscillations share more or less the same frequency.

\pagebreak
\begin{multicols}{2}
\subsection*{Relaxation time as function of $\gamma$}
\singlefigure{./plots/1_3/study.png}
To find relaxation time, I fit the exponential $f(t) = A e^{-Bt}$ to local peaks of the numerical simulation.
After finding values for $A$ and $B$, the relaxation time becomes $t_r = \frac{1}{B}$. The result is shown in the plot above.

The drop to exactly 0 around $\gamma\approx 5.9$ is exactly when the oscillator becomes critically dampened, as we only have one local maxima (the initial value) for such a system, and cannot fit the above mentioned exponential (that has two parameters) to a function with a single maximum. 
\columnbreak
\subsection*{Minimum x as function of $\gamma$}
\singlefigure{./plots/1_3/study_2.png}
This plot shows the minimum values of $\theta (t)$ over time for different $\gamma$. Just like in the previous section, we can conclude that the system becomes critically dampened around $\gamma \approx 5.9$ (even though it's harder to find the exact value in this plot).
\end{multicols}

\section*{1.4}
\subsection*{Phase space portrait of dampened pendulum}
\singlefigure{./plots/1_4/study.png}
What this phase space portrait shows us is that the oscillations will eventually die out, because the spiral converges towards $\theta(t_\infty) = 0$, $\theta'(t_\infty) = 0$.

\pagebreak
\section*{1.5}
\begin{multicols}{2}
  \subsection*{Energy average}
  \singlefigure{./plots/1_5/study.png}
  We see here that both the Leapfrog method and Velocity Verlet are energy preserving methods over long time, even for a relatively large time step as $\Delta t = 0.25$.
  \columnbreak
  \subsection*{Trajectories}
  \singlefigure{./plots/1_5/study2.png}
  Leapfrog produces different trajectory than Velocity Verlet, which is interesting to note.
\end{multicols}

% ![](./plots/1_1/comparison_1.png)
% ![](./plots/1_1/comparison_2.png)
% ![](./plots/1_1/comparison_3.png)
\end{document}