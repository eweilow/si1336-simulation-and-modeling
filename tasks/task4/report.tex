\documentclass[11pt]{article}
\usepackage[cm]{fullpage}
\usepackage{graphicx}
\usepackage{caption}
\usepackage{subcaption}
\usepackage[section]{placeins}
\usepackage{float}
\usepackage{amsmath}
\usepackage{multicol}

\setlength{\columnsep}{1cm}

\title{S1336 - Project 4}
\author{Erik Weilow}

\newcommand{\triplefigure}[3]{
\begin{figure}[H]
  \centering
  \begin{minipage}{0.3\textwidth}
    \centering
    \includegraphics[width=\textwidth]{#1}
  \end{minipage}
  \begin{minipage}{0.3\textwidth}
    \centering
    \includegraphics[width=\textwidth]{#2}
  \end{minipage}
  \begin{minipage}{0.3\textwidth}
    \centering
    \includegraphics[width=\textwidth]{#3}
  \end{minipage}
\end{figure}
}
\newcommand{\doublefigure}[2]{
\begin{figure}[H]
  \centering
  \begin{minipage}{0.45\textwidth}
    \centering
    \includegraphics[width=\textwidth]{#1}
  \end{minipage}
  \begin{minipage}{0.45\textwidth}
    \centering
    \includegraphics[width=\textwidth]{#2}
  \end{minipage}
\end{figure}
}
\newcommand{\singlefigure}[1]{
\begin{figure}[H]
  \centering
  \begin{minipage}{0.4\textwidth}
    \centering
    \includegraphics[width=\textwidth]{#1}
  \end{minipage}
\end{figure}
}
\newcommand{\singlewiderfigure}[1]{
\begin{figure}[H]
  \centering
  \begin{minipage}{0.6\textwidth}
    \centering
    \includegraphics[width=\textwidth]{#1}
  \end{minipage}
\end{figure}
}

\begin{document}
\maketitle
\newpage

\section*{4.1}
When looking at such a complex model (in the sense that any sort of error tends to break the results), it's nice to look at some sample solutions.
We can find some in the following plot:
\singlewiderfigure{./plots/4_1/sim1.png}
Each vehicle is plotted as a unique color, so we can easily verify that they don't collide or run around each other.
These simulations were run for a density $\rho = 20\%$ (10 vehicles on a road length of 50) with random starting positions, which is why the flow rate (that funnily turns out to be $Hz$) fluctuates slightly between simulations.

\section*{4.2}
This section sort of merged with 4.1, as I decided to plot density against flow rate for different road lengths in the same diagram.
\singlewiderfigure{./plots/4_2/density.png}

We can see that there's a clear dependence of road length up until at least a road length of around 50. There's also a clear point around $\rho \approx 30\%$ at which traffic jams start happening.

\section*{4.3}
In this section we're varying $v_{max}$ to see what effect it has on the behaviour of the vehicles.
\singlewiderfigure{./plots/4_3/sim1.png}

This plot was made using the probability of braking given for 4.1, $50\%$.
We can see from the above plots that the flow rate doesn't necessarily increase beyond $0.3 Hz$, even with increased maximum velocity, but the vehicles start running into each other more with increased maximum velocity.

\section*{4.4}
In this section we're varying the probability of decelleration to see what effect it has on the behaviour of the vehicles.
\singlewiderfigure{./plots/4_4/sim1.png}

These plots was made using a maximum velocity $v_{max} = 2$. Here we see that a lower probability of random braking increases flow and flow rate.
The conclusion one can draw from this is that one shouldn't be distracted when in traffic, since then we wouldn't have any traffic jams\footnote{One would also first need to prove that this simplified traffic model is a good representation of reality}

It makes sense to get the results shown in the plots, as if no braking occurs, the cars would eventually reach some form of equilibrium velocities.

\pagebreak

\section*{4.5}
We'll first look at a sample forest fire simulation (with a grid size of 8x8), where the squares in dark cyan are trees and the squares in yellow are trees on fire. 
We can see that the simulation follows the rules.
\singlewiderfigure{./plots/4_5/sample.png}

\subsection*{N(s) for $g = f = 10\%$}
Running 1000 simulation iterations on a grid size of 40x40 gives us the following histogram of sizes of the trees that are lit on fire in each iteration:
\singlewiderfigure{./plots/4_5/histogram.png}
The distribution seems to follow $N s^{-\alpha}$ quite well, with parameters $N \approx 18147$ and $\alpha \approx -1.61$ shown in the above figure acquired by fitting $N s^{-\alpha}$ to the histogram data.

\subsection*{N(s) for many g, f}
Every point in the two-dimensional interval $g \in [0.1, 0.5], f \in [0.1, 0.5]$ was averaged over around 14 simulations 
to produce the following plot of $\alpha$, acquired by fitting $N s^{-\alpha}$ to the histogram of each simulation, as a function of $g$ and $f$.
\singlewiderfigure{./plots/4_5/parameters.png}
It seems counterintuitive that increased growth probability would result in a smaller $\alpha$, as then $N(s)$ would grow smaller much faster.
One would think that higher growth would lead to larger clusters, so $\alpha$ should be smaller, but that's not the result that can be seen above.



\end{document}